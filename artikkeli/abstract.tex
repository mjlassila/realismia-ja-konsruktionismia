



\include{abstract}
%===================Abstract=======================
\newpage
\thispagestyle{empty}
\begin{abstract}
\begin{normalsize}
\noindent Kriittistä realismia ja sosiaalista konstruktionismia pidetään
yhteiskuntatieteissä usein kahtena keskenään kilpailevena
tieteenfilosofisena koulukuntana. Sosiaalityön kaltaisessa soveltavassa,
vahvasti laadullisen tutkimuksen suuntaan orientoituvassa tieteessä
sosiaalisesta konstruktionismista on tullut hyvin suosittu
tieteenfilosofia. Myös kriittisellä realismilla on kannattajansa. Tämän
artikkelin tarkoituksena on katsoa, miten sosiaaliseen
konstruktionismiin ja kriittiseen realismiin sitoutuneet opinnäytetyöt
eroavat toisistaan. Aineiston muodostaa Suomen ammattikorkeakoulujen
yhteisestä Theseus -opinnäytetietokannasta poimitut 21 sosiaalista
konstruktionismia ja 23 kriittistä realismia edustavaa
opinnäytetutkimusta. Aineistoa analysoidaan~ sekä laadullisen sisällön
analyysin että rakennemallinnuksen keinoin.~

\end{normalsize}
\end{abstract}

%===================Abstract=======================
\newpage
\thispagestyle{empty}
\begin{abstract}
\begin{normalsize}
\noindent Kriittistä realismia ja sosiaalista konstruktionismia pidetään
yhteiskuntatieteissä usein kahtena keskenään kilpailevena
tieteenfilosofisena koulukuntana. Sosiaalityön kaltaisessa soveltavassa,
vahvasti laadullisen tutkimuksen suuntaan orientoituvassa tieteessä
sosiaalisesta konstruktionismista on tullut hyvin suosittu
tieteenfilosofia. Myös kriittisellä realismilla on kannattajansa. Tämän
artikkelin tarkoituksena on katsoa, miten sosiaaliseen
konstruktionismiin ja kriittiseen realismiin sitoutuneet opinnäytetyöt
eroavat toisistaan. Aineiston muodostaa Suomen ammattikorkeakoulujen
yhteisestä Theseus -opinnäytetietokannasta poimitut 21 sosiaalista
konstruktionismia ja 23 kriittistä realismia edustavaa
opinnäytetutkimusta. Aineistoa analysoidaan~ sekä laadullisen sisällön
analyysin että rakennemallinnuksen keinoin.~

\end{normalsize}
\end{abstract}

%===================Abstract=======================
\newpage
\thispagestyle{empty}
\begin{abstract}
\begin{normalsize}
\noindent Kriittistä realismia ja sosiaalista konstruktionismia pidetään
yhteiskuntatieteissä usein kahtena keskenään kilpailevena
tieteenfilosofisena koulukuntana. Sosiaalityön kaltaisessa soveltavassa,
vahvasti laadullisen tutkimuksen suuntaan orientoituvassa tieteessä
sosiaalisesta konstruktionismista on tullut hyvin suosittu
tieteenfilosofia. Myös kriittisellä realismilla on kannattajansa. Tämän
artikkelin tarkoituksena on katsoa, miten sosiaaliseen
konstruktionismiin ja kriittiseen realismiin sitoutuneet opinnäytetyöt
eroavat toisistaan. Aineiston muodostaa Suomen ammattikorkeakoulujen
yhteisestä Theseus -opinnäytetietokannasta poimitut 21 sosiaalista
konstruktionismia ja 23 kriittistä realismia edustavaa
opinnäytetutkimusta. Aineistoa analysoidaan~ sekä laadullisen sisällön
analyysin että rakennemallinnuksen keinoin.~

\end{normalsize}
\end{abstract}

%===================Abstract=======================
\newpage
\thispagestyle{empty}
\begin{abstract}
\begin{normalsize}
\noindent Kriittistä realismia ja sosiaalista konstruktionismia pidetään
yhteiskuntatieteissä usein kahtena keskenään kilpailevena
tieteenfilosofisena koulukuntana. Sosiaalityön kaltaisessa soveltavassa,
vahvasti laadullisen tutkimuksen suuntaan orientoituvassa tieteessä
sosiaalisesta konstruktionismista on tullut hyvin suosittu
tieteenfilosofia. Myös kriittisellä realismilla on kannattajansa. Tämän
artikkelin tarkoituksena on katsoa, miten sosiaaliseen
konstruktionismiin ja kriittiseen realismiin sitoutuneet opinnäytetyöt
eroavat toisistaan. Aineiston muodostaa Suomen ammattikorkeakoulujen
yhteisestä Theseus -opinnäytetietokannasta poimitut 21 sosiaalista
konstruktionismia ja 23 kriittistä realismia edustavaa
opinnäytetutkimusta. Aineistoa analysoidaan~ sekä laadullisen sisällön
analyysin että rakennemallinnuksen keinoin.~

\end{normalsize}
\end{abstract}
