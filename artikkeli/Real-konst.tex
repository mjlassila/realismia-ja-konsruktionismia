\documentclass[a4paper,11pt,finnish]{article}
\usepackage[utf8]{inputenc}
\usepackage[T1]{fontenc}
\usepackage[]{natbib}
\usepackage[finnish]{babel}
\usepackage{setspace}
\usepackage{palatino}
\usepackage{marvosym}
\bibpunct{(}{)}{;}{a}{}{,}
\sloppy
\doublespacing
\usepackage[]{graphics}
%Usage: \btable{table specs}{caption}{reference label} 
\newcommand{\btable}[3]{ \begin{table}[htbp] \begin{center}
\caption{#2\label{#3}} \begin{tabular}{#1}} 
%
\newcommand{\etable}{\end{tabular} \end{center} \end{table}}

%\input{_epspdf_}
%\DeclareGraphicsExtensions{.pdf}
%\DeclareGraphicsExtensions{.ps,.eps}
%\usepackage[pdftitle={Otsikko},
%pdfauthor={Mikko Mäntysaari },
%colorlinks=true,linkcolor=red]{hyperref}

%\includeonly{muodot}
%%------------------------------------------------------
\author{Mikko M{\"a}ntysaari, Matti Lassila}
\title{Realismia ja konstruktionismia opinnäytteissä}
\begin{document}
\pagestyle{plain}
\maketitle
%%------------------------------------------------------
%\tableofcontents
%\listoffigures
%\listoftables

% Kappaleen ensimmäisen rivin sisennys pois ja
% kappaleiden väliin tyhjää.
%\setlength{\parindent}{0pt}
%\setlength{\parskip}{1ex plus 0.5ex minus 0.2ex}

% dokumentin alku:
%--------------------------------------------------------


%===================Abstract=======================
\newpage
\thispagestyle{empty}
\begin{abstract}
\begin{normalsize}
\noindent Kriittistä realismia ja sosiaalista konstruktionismia pidetään
yhteiskuntatieteissä usein kahtena keskenään kilpailevena
tieteenfilosofisena koulukuntana. Sosiaalityön kaltaisessa soveltavassa,
vahvasti laadullisen tutkimuksen suuntaan orientoituvassa tieteessä
sosiaalisesta konstruktionismista on tullut hyvin suosittu
tieteenfilosofia. Myös kriittisellä realismilla on kannattajansa. Tämän
artikkelin tarkoituksena on katsoa, miten sosiaaliseen
konstruktionismiin ja kriittiseen realismiin sitoutuneet opinnäytetyöt
eroavat toisistaan. Aineiston muodostaa Suomen ammattikorkeakoulujen
yhteisestä Theseus -opinnäytetietokannasta poimitut 21 sosiaalista
konstruktionismia ja 23 kriittistä realismia edustavaa
opinnäytetutkimusta. Aineistoa analysoidaan~ sekä laadullisen sisällön
analyysin että rakennemallinnuksen keinoin.~

\end{normalsize}
\end{abstract}

%================Begin Manuscript==================
\newpage






\section*{Johdanto}\label{johdanto}

Yhteiskuntatieteissä on aina yhtä aikaa monia keskenään kilpailevia
metodologisia koulukuntia. Joitakin kymmeniä vuosia sitten vallitsevia
tieteenfilosofisia koulukuntia olivat looginen empirismi tai
positivismi, marxismi ja fenomenologia ja muut mannermaisen filosofian
pohjalta syntyneet ajattelutavat. Koulukunnat vaihtuvat ja muuntuvat, ja
esimerkiksi marxismiin viittaavia sosiaalitieteellisiä opinnäytteitä ei
enää juuri näe. Tarkastelemme seuraavassa kahden viime vuosina asemansa
vakiinnuttaneen tieteenfilosofisen koulukunnan, sosiaalisen
konstruktionismin ja kriittisen realismin keskeisiä väittämiä lyhyesti
ja pääpiirteisesti. Sosiaalinen konstruktionismi on 1990-luvulta lähtien
voimistanut asemiaan eräänlaisena uutena normaalitieteenä ainakin
sosiaalitieteissä. Sille kilpailijaksi on noussut kriittisen realismin
tieteenfilosofia.

Viime vuosikymmeninä sosiaalisesta konstruktionismista on tullut ainakin
sosiaalityön kaltaisessa soveltavassa tutkimusalassa keskeinen
metodologinen suuntaus. Sosiaalisen konstruktionismin taustalla on
joukko toisiinssa liittyviä filosofisia perinteitä. Husserlin ja
Heideggerin fenomenologinen perinne, Schützin filosofia, Berger ja
Luckmann\ldots{}

Suomessa sosiaalinen konstruktionismi on liittynyt usein
diskurssianalyysiin. Ranskalainen poststrukturalismi ja erityisesti
Foucault ovat vaikuttaneet diskurssianalyysiksi kutsutun metodologisen
orientaation syntyyn. Sosiaalityön tutkimuksessa on tarkoin seurattu
toisten sosiaalitieteiden metodologista keskustelua. Erityisesti
tamperelaiset tutkijat professori Kirsi Juhilan ja professori Arja
Jokisen johdolla ovat julkaisseet runsaasti sekä tutkimusta että
diskurssianalyysin tekemiseen perehdyttäviä oppikirjoja, jotka ovat
saaneet runsaasti lukijoita.

Filosofiassa erilaisissa realismin muodoilla on pitkä, aina Platoniin ja
Aristoteleeseen ulottuva historia. Erilaisia realismin variantteja on
olemassa lukuisia. Tieteenfilosofisessa kirjallisuudessa puhutaan paljon
tieteellisestä realismista, jonka eri variantteja Suomessa ovat
kannattaneet vaikkapa Ilkka Niiniluoto ja Raimo Tuomela. Tuoreessa
keskustelukirjassaan Manuel Delanda and Graham Haarman
\citep{Delanda_Harman_2017} keskustelevat realismin merkityksen kasvusta
filosofiassa. Heidän mukaansa filosofian tutkimuksessa on viime aikoina
herännyt yhä enemmän kiinnostusta realistista ajattelua kohtaan.

Realismiin liitetään usein kaksi näkökulmaa: toinen argumentti liittyy
maailman olemassaoloon, toinen siihen, että maailma on olemassa meistä
riippumatta \cite{sep-realism}.

Sosiaalitieteissä suhde realismiin on muuttunut erityisesti kriittisen
realismin myötä. Noin neljäkymmentä vuotta sitten brittiläinen
yhteiskuntatieteilijä Roy Bhaskar kehitti tieteenfilosofisen kannan,
jota hän itse kutsui ``transendentaaliseksi realismiksi'' -- muotoilun
takana oli tietenkin Kantin transendentaali idealismi. Bhaskar erotti
tieteenfilosofian historiassa kolme suurta linjaa:

\begin{enumerate}

\item
  Klassisen empirismin, jota luonnehtii atomististen tapahtumien
  pitäminen tiedostuksen kohteena;
\item
  Transsendentaalinen idealismi, jonka mukaan tieteellisen tiedon
  kohteena ovat todellisuuden olemusta käsittelevät ideat ja mallit,
  joita ei pidetä ihmisestä riippumattomina;
\item
  Transsendentaalinen realismi, jonka mukaan tiedon kohteena ovat
  rakenteet ja mekanismit, jotka luovat ilmiömaailman. Tämä tieto syntyy
  tieteen avulla. Rakenteet jotka synnyttävät ilmiöt, ovat olemassa
  tiedostamme riippumatta. \citep[][277]{Mantysaari_1991}.
\end{enumerate}

Bhaskar määritteli transsendentaalin realismin näin:

\begin{quote}
``It regards the objects of knowledge as structures and mechanisms that
generate phenomena; and the knowledge as produced in the social activity
of science.'' \citep{Bhaskar_1975}.
\end{quote}

Sittemmin Bhaskarin transsendentaalisesta realismista tuli kriittistä
realismia. Varsinainen kriittisen realismin ekspansio tapahtui
1990-luvulla, ei vähiten siksi, että Pawson ja Tilley julkaisivat
kriittisen realismin perustalle rakentuvan arviointitutkimusta
käsittelevän teoksensa \citep{RefWorks:1045}.

Professori Vuokko Niiranen on yhdessä kollegojensa kanssa toimittanut
kirjan \emph{Realismin haaste yhteiskuntatieteissä} \citep{Solr-jykdok.1004015}.
Teos on edelleen ainoa suomenkielinen kriittistä realismia tarkasteleva
teos. Vuokko Niiranen tarkastelee omassa artikkelissaan johtamisen
kausaalisia voimia ja mekanismeja. \citep{niiranen2006} Niiranen hakee
kriittisen realismin yhteyksiä organisaatiotutkimuksen ja johtamisen
tutkimisen pitkään traditioon ja suhtautuu mielestämme terveen
kriittisesti kriittisen realismin ainutlaatuisuutta ja täydellistä
uutuutta koskeviin väitteisiin. Vaikkapa resurssiriippuvuusteorian ja
kriittisen realismin väliltä näyttäisi löytyvän yhtäläisyyksiä.

\par\null

\section*{Tutkimustehtävä}

Tarkoituksemme on tutkia, miten sosiaalialan opiskelijat käyttävät
kriittisen realismin ja sosiaalisen konstruktionismin käsitteitä
opinnäytetöissään, millaisia tutkimusteemoja opinnäytteissä esiintyy, ja
miten tutkimukset eroavat toisistaan. Kriittisen realismin ja
sosiaalisen konstruktionismin vastakkaiset ontologiset kannat (realismi
ja idealismi) ja niistä aiheutuvat erimielisyydet epistomologiassa ja
siitä johtuen tavassa nähdä tutkimuksen metodiset kysymykset, voisi
olettaa näkyvän opiskelijoiden opinnäytetöissä. Myös lähteiden voisi
olettaa eroavan.

Hypoteesi on, että sosiaalisen konstruktionismin ja kriittisen realismin
näkökulmista tehdyt opinnäytetyöt eroavat sisällöltään toisistaan.

\section*{Aineisto ja menetelmät}

Alun perin tarkoitus oli tutkia sosiaalityön maisteritutkintojen
opinnäytteitä, eli pro gradu-tutkielmia. Vaikka yliopistojen
opinnäytteitä on mahdollista selata XXX käyttöliittymän avulla,
osoittautui, että eri yliopistojen opinnäytteisiin ei voida tehdä
helposti sisältöön kohdistuvia hakuja. Hakuja voidaan suunnata
tutkielmien nimiin ja asiasanoihin (abstrakteihin?), mutta ei
tutkielmien sisältöön.

Näiden hankaluuksien vuoksi päädyimme käyttämään suomalaisten
ammattikorkeakoulujen yhteisä Theseus tutkielmatietokantaa. Theseuksessa
on kaikkien ammattikorkeakoulujen kaikki tutkielmat pdf-muodossa
vapaasti saatavana. Tutkielmien nimiin, asiasanoihin ja myös sisältöihin
(ensimmäiseen 3000 sanaan) voidaan suunnata hakuja. Theseuksessa on jo
noin 120 000 tutkielmaa. Ammattikorkeakoulututkintoihin sisältyvät
opinnäytteet voivat olla joko alemman korkeakoulututkinnon tasoisia tai
sitten ylempään ammattikorkeakoulututkintoon sisältyviä. Tässä
aineistossa emme ole tehneet eroa eri ammattikorkeakoulututkintojen
välillä. Ylemmän amk-tutkinnon opinnäytteet ovat yleensä selvästi
alemman tutkinnon tutkielmia sivumääräisesti laajempia ja niiden
``tieteellinen kunnianhimo'' on korkeammalla tasolla. Yliopistojen
sosiaalitieteiden ja myös sosiaalityön opintoihin sisältyy huomattavasti
enemmän tutkimusmenetelmien opetusta, joten pro gradu tutkielmissa
käsitellään menetelmävalintoja selvästi painokkaammin kuin
keskimääräisessä amk-tutkielmassa on mahdollista. Koska tutkimuksemme
kohdistuu ennen muuta tapoihin, joilla konstruktionistisia ja
realistisia metodologioita käytetään tutkimuksessa hyväksi, ei
amk-tutkielmien rajoitetumpi metodologinen tarkastelu ole aineiston
kannalta kovin vakava ongelma.

Ammattikorkeakoulujen tutkielmia ei ole välttämättä kirjoitettu
sosiaalityön tutkimuksen viitekehyksessä. Monet amk-opiskelijat
suuntautuvat varhaiskasvatuksen tehtäviin, jolloin tutkielmien
viitekehys on lähinnä kasvatustieteellinen. Silti suuri osa tutkielmista
on näkökulmaltaan selvästi sosiaalityön tutkimusta. Tämä ei ole sinänsä
ihme, sillä tutkielmia ohjaavat opettajat saattavat olla sosiaalityössä
väitelleitä. Varsinkin ylempien amk-tutkielmien voi tasoltaan ajatella
hyvin vastaavan keskimääräistä yliopistollista pro gradu tutkielmaa.
Ammattikorkeakouluopinnäytteisiin keskittymistä voi puolustaa myös siitä
näkökulmasta, että amk-tutkielmissa on paljaana näkyvissä opittu
perusnäkemys siitä, miten empiiristä tutkimusta ajatellaan tehtäväksi.
Vain harvalla amk-opiskelijalla on ollut mahdollisuus perehtyä kovin
syvällisesti metodologisiin ja filosofisiin keskusteluihin.

Toukokuussa 2018 tehdyissä hauissa käytettiin seuraavia hakulausekkeita,
ja haun osoittamat tutkielmat ladattiin pdf muodossa lähempää
tarkastelua varten:

Kriittinen realismi:

\begin{quote}
(``kriittinen reali*" OR ``realistinen eval\emph{" OR ``realistic
evaluation''\textasciitilde{}3 OR ``kriittinen
realismi''\textasciitilde{}3 OR ``realistinen
evaluointi''\textasciitilde{}3 OR ``realistinen
arviointi''\textasciitilde{}2) AND programme:sosiaali} AND NOT
programme:terveys*
\end{quote}

Sosiaalinen konstruktionismi:

\begin{quote}
(``sosiaalinen konstruktionismi''\textasciitilde{}3 OR
sosiaalikonstruktionismi OR ``sosiaalinen
konstruoituminen''\textasciitilde{}3 OR relativismi OR
diskurssianalyy\emph{) AND programme:sosiaali} AND NOT
programme:terveys*
\end{quote}

Kriittinen realismi -- haku palautti 9.5.2018 yhteensä 65 osumaa, joista
poistettiin 42 tutkielmaa, joissa kyse ei ollut kriittisestä realismista
vaan sanojen ``kriittinen'' ja ``realismi'' esiintymisestä riittävän
lähellä toisiaan. Jäljelle jäi 23 tutkielmaa, jossa oli käsitelty
kriittistä realismia tai realistista arviointitutkimusta. Sosiaalinen
konstruktionismi -- haku palautti 9.5.2018 329 osumaa, eli huomattavasti
enemmän kuin kriittistä realismia koskeva haku. Koska haun tuottamia
osumia oli näin runsaasti, kävimme tarkastelemaan vain ensimmäistä 100.
haun tuottamaa tutkielmaa. Osoittautui, että 79:ssa tutkielmassa
sosiaalista konstruktionismia ja/tai diskurssianalyysia oli käsitelty
aivan suppeasti. Tavallisesti saattoi tutkielmassa olla vain viittaus
Eskolan ja Suorannan menetelmäteokseen, jossa oli lueteltu erilaisia
laadullisen tutkimuksen suuntauksia, yhtenä näistä diskurssianalyysi.
Tarkastelun jälkeen jälkelle jäi 21 sosiaalisen konstruktionismiin
(diskurssianalyysiin) kiinnittynyttä työtä.

Laadullista sisällön analyysia varten valitut 43 tutkielmaa
tallennettiin pdf-muodossa ensiksi \emph{Atlas ti}-laadullisen aineiston
analyysiohjelmaan. Tämän jälkeen opinnäytetöitä koodattiin erityisesti
totuusteorioihin, tuloksellisuuteen ja vaikuttavuuteen liittyvien
tekstijaksojen löytämiseksi.

Tämän jälkeen valituista tutkielmista tallennettiin tekstimuodossa
kappale, joista poistettiin kansilehdet, abstraktit, sisällysluettelot,
lähdeluettelot ja lähteiden jälkeiset liitteet. Tämän jälkeen teksit
perusmuotoistettiin (``lemmatisaatio'').

Aineistoa käsiteltiin RStudio - tilasto-ohjelmaan sisältyvien
rakennemallinnukseen (structural modelling) tekstianalyysityökalujen
avulla. Stop-sanojen poistolla tarkastelun ulkopuolelle rajautuivat
tyypilliset suomenkieliset sanat, joilla ei ajateltu olevan aineiston
analyysin kannalta merkitystä.

Viimeiseksi toteutettiin tutkielmien lähdeaineistoon kohdistuva
tarkastelu (\ldots{})

Rakennemallinnusta ei tietääksemme ole aikaisemmin hyödynnetty
sosiaalityön tutkimuksessa, ei ainakaan Suomessa. Kun nk. datalouhinta
suurilla aineistoilla on tulossa entistä tärkeämmäksi osaksi
yhteiskuntatieteiden menetelmäarsenaalia, on näitä menetelmiä ryhdyttävä
ennakkoluulottomasti soveltamaan myös sosiaalityön tutkimuksessa.

Aineiston analyysi on toteutettu niin, että Mikko Mäntysaari on
vastannut laadullisesta sisällönanalyysista ja Matti Lassila
rakennemallinnuksesta.

\section*{Tulokset}

\subsection*{Laadulliset tulokset}

\emph{Atlas ti-} ohjelman avulla tuotettu tekstien koodaus tuotti
tavallaan selkeän tuloksen: konstruktivistis/diskurssianalyyttiset ja
realistiset työt erosivat sisällöltään toisistaan erityisesti
tutkimusmenetelmien kuvauksessa. Konstruktivistisiin töihin liittyi
usein episteemisen relativismin mukainen tarina siitä, kuinka
\emph{totuutta ei ole olemassa}, ja että todellisuutta koskevat
tulkinnat ovat kieleen sidottuja. Opinnäytteissä käytettävä
argumentaatio on hyvin samankaltaista kuin yliopistojen pro
gradututkielmissa. Vastaavasti realistisissa töissä viitattiin usein nk.
\emph{CMO-kaavaan}, jonka mukaan projektin tai intervention
vaikuttavuutta voidaan tarkastella vain kontekstissaan.

\subsection*{Rakennemallinnuksen tulokset}



\section*{Johtopäätökset}




% dokumentin loppu
%--------------------------------------------------------
\bibliography{kirjat}
\bibliographystyle{apa}
\end{document}


