\section*{Johdanto}

Yhteiskuntatieteissä on aina yhtä aikaa monia keskenään kilpailevia
metodologisia koulukuntia. Joitakin kymmeniä vuosia sitten vallitsevia
tieteenfilosofisia koulukuntia olivat looginen empirismi tai
positivismi, marxismi ja fenomenologia ja muut mannermaisen filosofian
pohjalta syntyneet ajattelutavat. Koulukunnat vaihtuvat ja muuntuvat, ja
esimerkiksi marxismiin viittaavia sosiaalitieteellisiä opinnäytteitä ei
enää juuri näe. Tarkastelemme seuraavassa kahden viime vuosina asemansa
vakiinnuttaneen tieteenfilosofisen koulukunnan, sosiaalisen
konstruktionismin ja kriittisen realismin keskeisiä väittämiä lyhyesti
ja pääpiirteisesti. Sosiaalinen konstruktionismi on 1990-luvulta lähtien
voimistanut asemiaan eräänlaisena uutena normaalitieteenä ainakin
sosiaalitieteissä. Sille kilpailijaksi on noussut kriittisen realismin
tieteenfilosofia.

Viime vuosikymmeninä sosiaalisesta konstruktionismista on tullut ainakin
sosiaalityön kaltaisessa soveltavassa tutkimusalassa keskeinen
metodologinen suuntaus. Sosiaalisen konstruktionismin taustalla on
joukko toisiinssa liittyviä filosofisia perinteitä. Husserlin ja
Heideggerin fenomenologinen perinne, Schützin filosofia, Berger ja
Luckmann\ldots{}

Suomessa sosiaalinen konstruktionismi on liittynyt usein
diskurssianalyysiin. Ranskalainen poststrukturalismi ja erityisesti
Foucault ovat vaikuttaneet diskurssianalyysiksi kutsutun metodologisen
orientaation syntyyn. Sosiaalityön tutkimuksessa on tarkoin seurattu
toisten sosiaalitieteiden metodologista keskustelua. Erityisesti
tamperelaiset tutkijat professori Kirsi Juhilan ja professori Arja
Jokisen johdolla ovat julkaisseet runsaasti sekä tutkimusta että
diskurssianalyysin tekemiseen perehdyttäviä oppikirjoja, jotka ovat
saaneet runsaasti lukijoita.

Filosofiassa erilaisissa realismin muodoilla on pitkä, aina Platoniin ja
Aristoteleeseen ulottuva historia. Erilaisia realismin variantteja on
olemassa lukuisia. Tieteenfilosofisessa kirjallisuudessa puhutaan paljon
tieteellisestä realismista, jonka eri variantteja Suomessa ovat
kannattaneet vaikkapa Ilkka Niiniluoto ja Raimo Tuomela. Tuoreessa
keskustelukirjassaan Manuel Delanda and Graham Haarman
\citep{Delanda_Harman_2017} keskustelevat realismin merkityksen kasvusta
filosofiassa. Heidän mukaansa filosofian tutkimuksessa on viime aikoina
herännyt yhä enemmän kiinnostusta realistista ajattelua kohtaan.

Realismiin liitetään usein kaksi näkökulmaa: toinen argumentti liittyy
maailman olemassaoloon, toinen siihen, että maailma on olemassa meistä
riippumatta \cite{sep-realism}.

Sosiaalitieteissä suhde realismiin on muuttunut erityisesti kriittisen
realismin myötä. Noin neljäkymmentä vuotta sitten brittiläinen
yhteiskuntatieteilijä Roy Bhaskar kehitti tieteenfilosofisen kannan,
jota hän itse kutsui ``transendentaaliseksi realismiksi'' -- muotoilun
takana oli tietenkin Kantin transendentaali idealismi. Bhaskar erotti
tieteenfilosofian historiassa kolme suurta linjaa:

\begin{enumerate}

\item
  Klassisen empirismin, jota luonnehtii atomististen tapahtumien
  pitäminen tiedostuksen kohteena;
\item
  Transsendentaalinen idealismi, jonka mukaan tieteellisen tiedon
  kohteena ovat todellisuuden olemusta käsittelevät ideat ja mallit,
  joita ei pidetä ihmisestä riippumattomina;
\item
  Transsendentaalinen realismi, jonka mukaan tiedon kohteena ovat
  rakenteet ja mekanismit, jotka luovat ilmiömaailman. Tämä tieto syntyy
  tieteen avulla. Rakenteet jotka synnyttävät ilmiöt, ovat olemassa
  tiedostamme riippumatta. \citep[][277]{Mantysaari_1991}.
\end{enumerate}

Bhaskar määritteli transsendentaalin realismin näin:

\begin{quote}
``It regards the objects of knowledge as structures and mechanisms that
generate phenomena; and the knowledge as produced in the social activity
of science.'' \citep{Bhaskar_1975}.
\end{quote}

Sittemmin Bhaskarin transsendentaalisesta realismista tuli kriittistä
realismia. Varsinainen kriittisen realismin ekspansio tapahtui
1990-luvulla, ei vähiten siksi, että Pawson ja Tilley julkaisivat
kriittisen realismin perustalle rakentuvan arviointitutkimusta
käsittelevän teoksensa \citep{RefWorks:1045}.

Professori Vuokko Niiranen on yhdessä kollegojensa kanssa toimittanut
kirjan \emph{Realismin haaste yhteiskuntatieteissä} \citep{Solr-jykdok.1004015}.
Teos on edelleen ainoa suomenkielinen kriittistä realismia tarkasteleva
teos. Vuokko Niiranen tarkastelee omassa artikkelissaan johtamisen
kausaalisia voimia ja mekanismeja. \citep{niiranen2006} Niiranen hakee
kriittisen realismin yhteyksiä organisaatiotutkimuksen ja johtamisen
tutkimisen pitkään traditioon ja suhtautuu mielestämme terveen
kriittisesti kriittisen realismin ainutlaatuisuutta ja täydellistä
uutuutta koskeviin väitteisiin. Vaikkapa resurssiriippuvuusteorian ja
kriittisen realismin väliltä näyttäisi löytyvän yhtäläisyyksiä.

\par\null
