
\section*{Aineisto ja menetelmät}

Alun perin tarkoitus oli tutkia sosiaalityön maisteritutkintojen
opinnäytteitä, eli pro gradu-tutkielmia. Vaikka yliopistojen
opinnäytteitä on mahdollista selata yliopistojen yhteisen
% mikä sen nimi on?
käyttöliittymän avulla,
osoittautui, että eri yliopistojen opinnäytteisiin ei voida tehdä
helposti sisältöön kohdistuvia hakuja. Hakuja voidaan suunnata
tutkielmien nimiin ja asiasanoihin (abstrakteihin?), mutta ei
tutkielmien sisältöön.

Näiden hankaluuksien vuoksi päädyimme käyttämään suomalaisten
ammattikorkeakoulujen yhteisä Theseus tutkielmatietokantaa. Theseuksessa
on kaikkien ammattikorkeakoulujen kaikki tutkielmat pdf-muodossa
vapaasti saatavana. Tutkielmien nimiin, asiasanoihin ja myös sisältöihin
(ensimmäiseen 3000 sanaan) voidaan suunnata hakuja. Theseuksessa on jo
noin 120 000 tutkielmaa. Ammattikorkeakoulututkintoihin sisältyvät
opinnäytteet voivat olla joko alemman korkeakoulututkinnon tasoisia tai
sitten ylempään ammattikorkeakoulututkintoon sisältyviä. Tässä
aineistossa emme ole tehneet eroa eri ammattikorkeakoulututkintojen
välillä. Ylemmän amk-tutkinnon opinnäytteet ovat yleensä selvästi
alemman tutkinnon tutkielmia sivumääräisesti laajempia ja niiden
``tieteellinen kunnianhimo'' on korkeammalla tasolla. Yliopistojen
sosiaalitieteiden ja myös sosiaalityön opintoihin sisältyy huomattavasti
enemmän tutkimusmenetelmien opetusta, joten pro gradu tutkielmissa
käsitellään menetelmävalintoja selvästi painokkaammin kuin
keskimääräisessä amk-tutkielmassa on mahdollista. Koska tutkimuksemme
kohdistuu ennen muuta tapoihin, joilla konstruktionistisia ja
realistisia metodologioita käytetään tutkimuksessa hyväksi, ei
amk-tutkielmien rajoitetumpi metodologinen tarkastelu ole aineiston
kannalta kovin vakava ongelma.

Ammattikorkeakoulujen tutkielmia ei ole välttämättä kirjoitettu
sosiaalityön tutkimuksen viitekehyksessä. Monet amk-opiskelijat
suuntautuvat varhaiskasvatuksen tehtäviin, jolloin tutkielmien
viitekehys on lähinnä kasvatustieteellinen. Silti suuri osa tutkielmista
on näkökulmaltaan selvästi sosiaalityön tutkimusta. Tämä ei ole sinänsä
ihme, sillä tutkielmia ohjaavat opettajat saattavat olla sosiaalityössä
väitelleitä. Varsinkin ylempien amk-tutkielmien voi tasoltaan ajatella
hyvin vastaavan keskimääräistä yliopistollista pro gradu tutkielmaa.
Ammattikorkeakouluopinnäytteisiin keskittymistä voi puolustaa myös siitä
näkökulmasta, että amk-tutkielmissa on paljaana näkyvissä opittu
perusnäkemys siitä, miten empiiristä tutkimusta ajatellaan tehtäväksi.
Vain harvalla amk-opiskelijalla on ollut mahdollisuus perehtyä kovin
syvällisesti metodologisiin ja filosofisiin keskusteluihin.

Toukokuussa 2018 tehdyissä hauissa käytettiin seuraavia hakulausekkeita,
ja haun osoittamat tutkielmat ladattiin pdf muodossa lähempää
tarkastelua varten:

Kriittinen realismi:

\begin{quote}
(``kriittinen reali*" OR ``realistinen eval\emph{" OR ``realistic
evaluation''\textasciitilde{}3 OR ``kriittinen
realismi''\textasciitilde{}3 OR ``realistinen
evaluointi''\textasciitilde{}3 OR ``realistinen
arviointi''\textasciitilde{}2) AND programme:sosiaali} AND NOT
programme:terveys*
\end{quote}

Sosiaalinen konstruktionismi:

\begin{quote}
(``sosiaalinen konstruktionismi''\textasciitilde{}3 OR
sosiaalikonstruktionismi OR ``sosiaalinen
konstruoituminen''\textasciitilde{}3 OR relativismi OR
diskurssianalyy\emph{) AND programme:sosiaali} AND NOT
programme:terveys*
\end{quote}

Kriittinen realismi -- haku palautti 9.5.2018 yhteensä 65 osumaa, joista
poistettiin 42 tutkielmaa, joissa kyse ei ollut kriittisestä realismista
vaan sanojen ``kriittinen'' ja ``realismi'' esiintymisestä riittävän
lähellä toisiaan. Jäljelle jäi 23 tutkielmaa, jossa oli käsitelty
kriittistä realismia tai realistista arviointitutkimusta. Sosiaalinen
konstruktionismi -- haku palautti 9.5.2018 329 osumaa, eli huomattavasti
enemmän kuin kriittistä realismia koskeva haku. Koska haun tuottamia
osumia oli näin runsaasti, kävimme tarkastelemaan vain ensimmäistä 100.
haun tuottamaa tutkielmaa. Osoittautui, että 79:ssa tutkielmassa
sosiaalista konstruktionismia ja/tai diskurssianalyysia oli käsitelty
aivan suppeasti. Tavallisesti saattoi tutkielmassa olla vain viittaus
Eskolan ja Suorannan menetelmäteokseen, jossa oli lueteltu erilaisia
laadullisen tutkimuksen suuntauksia, yhtenä näistä diskurssianalyysi.
Tarkastelun jälkeen jälkelle jäi 21 sosiaalisen konstruktionismiin
(diskurssianalyysiin) kiinnittynyttä työtä.

Laadullista sisällön analyysia varten valitut 43 tutkielmaa
tallennettiin pdf-muodossa ensiksi \emph{Atlas ti}-laadullisen aineiston
analyysiohjelmaan. Tämän jälkeen opinnäytetöitä koodattiin erityisesti
totuusteorioihin, tuloksellisuuteen ja vaikuttavuuteen liittyvien
tekstijaksojen löytämiseksi.

Tämän jälkeen valituista tutkielmista tallennettiin tekstimuodossa
kappale, joista poistettiin kansilehdet, abstraktit, sisällysluettelot,
lähdeluettelot ja lähteiden jälkeiset liitteet. Tämän jälkeen teksit
perusmuotoistettiin (``lemmatisaatio'').

Aineistoa käsiteltiin RStudio - tilasto-ohjelmaan sisältyvien
rakennemallinnukseen (structural modelling) tekstianalyysityökalujen
avulla. Stop-sanojen poistolla tarkastelun ulkopuolelle rajautuivat
tyypilliset suomenkieliset sanat, joilla ei ajateltu olevan aineiston
analyysin kannalta merkitystä.

Viimeiseksi toteutettiin tutkielmien lähdeaineistoon kohdistuva
tarkastelu (\ldots{})

Rakennemallinnusta ei tietääksemme ole aikaisemmin hyödynnetty
sosiaalityön tutkimuksessa, ei ainakaan Suomessa. Kun nk. datalouhinta
suurilla aineistoilla on tulossa entistä tärkeämmäksi osaksi
yhteiskuntatieteiden menetelmäarsenaalia, on näitä menetelmiä ryhdyttävä
ennakkoluulottomasti soveltamaan myös sosiaalityön tutkimuksessa.

Analyysissa hyödynnettiin tietokoneavusteista ainealueiden tunnistusta
(``topic extraction''). Tällä metodologialla pyritään
\emph{semanttiseen} analyysiin. Vaikka tietokone ei luonnollisestikaan
kykene ymmärtämään tekstin merkityksiä samalla varmuudella kuin
ihmiset, kyetään moderneilla semanttisen analyysin välineillä varsin
hyvään tarkkuuteen. Erityisesti suurten tekstimassojen käsittelyssä
tietokoneista voi olla paljon hyötyä. Tilastollisia menetelmiä
hyödyntäen suuresta tekstimassasta voidaan etsiä keskeisiä aihealueita
varsin luotettavasti. Aihealueiden tunnistus pyrkii tekstin
merkistysten osoittamiseen, ei vain sanojen esiintyvyyden
tutkimiseen. \citep[57-59.]{leetaru}

Aineiston analyysi on toteutettu niin, että Mikko Mäntysaari on
vastannut laadullisesta sisällönanalyysista ja Matti Lassila
rakennemallinnuksesta.

