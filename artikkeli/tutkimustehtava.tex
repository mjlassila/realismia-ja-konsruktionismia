\section*{Tutkimustehtävä}

Tarkoituksemme on tutkia, miten sosiaalialan opiskelijat käyttävät
kriittisen realismin ja sosiaalisen konstruktionismin käsitteitä
opinnäytetöissään, millaisia tutkimusteemoja opinnäytteissä esiintyy, ja
miten tutkimukset eroavat toisistaan. Kriittisen realismin ja
sosiaalisen konstruktionismin vastakkaiset ontologiset kannat (realismi
ja idealismi) ja niistä aiheutuvat erimielisyydet epistomologiassa ja
siitä johtuen tavassa nähdä tutkimuksen metodiset kysymykset, voisi
olettaa näkyvän opiskelijoiden opinnäytetöissä. Myös lähteiden voisi
olettaa eroavan.

Hypoteesi on, että sosiaalisen konstruktionismin ja kriittisen realismin
näkökulmista tehdyt opinnäytetyöt eroavat sisällöltään toisistaan.
